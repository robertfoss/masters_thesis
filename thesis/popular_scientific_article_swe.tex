\documentclass[12pt,a4paper,onecolumn,openright]{report}

\usepackage{thesis}
\usepackage{float}
\usepackage{amssymb}
\usepackage{multicol}
\usepackage{adjustbox}


\begin{document}

\begin{center}
\begin{spacing}{1.5}
\textbf{\Large Basic-block vektorisering för grafik-kompilatorer}
\end{spacing}
\end{center}

\begin{multicols}{2}
Grafikprocessorer är byggda för att hantera parallella uppgifter, som att beräkna färgen för varje pixel på en skärm. Denna parallellism är bara användbar om den effektivt utnyttjas och ett vanligt tillvägagångssätt är vektorisering.

Vektorisering är ett sätt att utföra två beräkningar vid ett och samma tillfälle. Till exempel om du beräknar $a + b$ och senare $c + d$, om de två beräkningarna kan utföras samtidigt istället för en åt gången, så har prestandan effektivt fördublats.


\begin{figure}[H]
\label{plot:w_reg}
\begin{tikzpicture}
    \begin{axis}[
		width  = 6.5cm,
		height = 4.5cm,
		major x tick style = transparent,
%		ybar=2*\pgflinewidth,	% No space between bars, use when bar borders are overlapping
		ybar=0pt,				% No space between bars, use if ybar=2*\pgflinewidth is causing issues
		bar width=14pt,
		ymin=0.95,
		ymajorgrids = true,
		major grid style={black!50},
		yminorgrids = true,
		minor y tick num={1},	% Set how many minor ticks to use per major tick
		minor grid style={gray!25},
		ylabel = {Registeranvändning},
		symbolic x coords={TestA, TestB, TestC},
		xtick = data,			% Prevent x-ticks from being spammed.
		scaled y ticks = false,
		enlarge x limits=0.25,	% Make x-axis larger to properly fit goups
		legend cell align=left,
		legend style={
			column sep=1ex,		% Separation between symbol and text
			anchor=north,
			at={(0.5,-0.25)},
			legend columns=-1	% Present legend in a single line
%			at={(1,1.05)},
%			anchor=south east
		}
	]
        \addplot[style={bblue,fill=bblue,mark=none}]
			coordinates {(TestA,1.00) (TestB,1.00) (TestC,1.00)};

        \addplot[style={rred,fill=rred,mark=none}]
			coordinates {(TestA,0.992) (TestB,1.00) (TestC,0.97)};


        \legend{Ingen vektorisering, Par-baserad}
    \end{axis}
\end{tikzpicture}
\end{figure}

Registeranvändning är ett sätt att mäta utrymme. Se register som RAM i din laptop eller mobiltelefon, men integrerad inuti en processor. För grafikprocessorer översätter minskad registeranvänding till ökad prestanda.

Resultaten som visas i registeranvändnings-diagrammet visar att prestandan ökar mellan +0.8\% och +3\%, och genomsnittligt +1.9\%.

\begin{figure}[H]
\label{plot:a_cycles}
\begin{tikzpicture}
    \begin{axis}[
		width  = 6.5cm,
		height = 4.5cm,
		major x tick style = transparent,
%		ybar=2*\pgflinewidth,	% No space between bars, use when bar borders are overlapping
		ybar=0pt,				% No space between bars, use if ybar=2*\pgflinewidth is causing issues
		bar width=14pt,
		ymin=0.95,
		ymajorgrids = true,
		major grid style={black!50},
		yminorgrids = true,
		minor y tick num={1},	% Set how many minor ticks to use per major tick
		minor grid style={gray!25},
		ylabel = {Aritmetiska cykler},
		symbolic x coords={TestA, TestB, TestC},
		xtick = data,			% Prevent x-ticks from being spammed.
		scaled y ticks = false,
		enlarge x limits=0.25,	% Make x-axis larger to properly fit goups
		legend cell align=left,
		legend style={
			column sep=1ex,		% Separation between symbol and text
			anchor=north,
			at={(0.35,-0.25)},
			legend columns=-1	% Present legend in a single line
%			at={(1,1.05)},
%			anchor=south east
		}
	]
        \addplot[style={bblue,fill=bblue,mark=none}]
			coordinates {(TestA,1.00) (TestB,1.00) (TestC,1.00)};

        \addplot[style={rred,fill=rred,mark=none}]
			coordinates {(TestA,1.014) (TestB,1.00) (TestC,0.997)};


%        \legend{No vectorization, Pair-based}
    \end{axis}
\end{tikzpicture}
\end{figure}

Aritmetiska cykler är ett mått på tiden det tar att utföra en beräkning. Cykler per sekund mäts i vad som vanligtvis kallas MHz.

Resultaten som visas i diagrammet för aritmetiska cykler visar att prestandan förändras mellan -1.4\% och +0.03\%, och genomsnittligt -0.69\%.


\begin{figure}[H]
\label{plot:cmp_time}
\begin{tikzpicture}
    \begin{axis}[
		width  = 6.5cm,
		height = 4.5cm,
		major x tick style = transparent,
%		ybar=2*\pgflinewidth,	% No space between bars, use when bar borders are overlapping
		ybar=0pt,				% No space between bars, use if ybar=2*\pgflinewidth is causing issues
		bar width=14pt,
		ymin=0.98,
		ymajorgrids = true,
		major grid style={black!50},
		yminorgrids = true,
		minor y tick num={1},	% Set how many minor ticks to use per major tick
		minor grid style={gray!25},
		ylabel = {Kompileringstid},
		symbolic x coords={TestA, TestB, TestC},
		xtick = data,			% Prevent x-ticks from being spammed.
		scaled y ticks = false,
		enlarge x limits=0.25,	% Make x-axis larger to properly fit goups
		legend cell align=left,
		legend style={
			column sep=1ex,		% Separation between symbol and text
			anchor=north,
			at={(0.5,-0.25)},
			legend columns=-1	% Present legend in a single line
%			at={(1,1.05)},
%			anchor=south east
		}
	]
        \addplot[style={bblue,fill=bblue,mark=none}]
			coordinates {(TestA,1.00) (TestB,1.00) (TestC,1.00)};

        \addplot[style={rred,fill=rred,mark=none}]
			coordinates {(TestA,0.988867306) (TestB,1.008398639) (TestC,0.995989841)};

%        \legend{No vectorization, Pair-based}
    \end{axis}
\end{tikzpicture}
\end{figure}


Processen där program som utvecklare skrivit översätts till något som en processor kan köra kallas kompilering. Det är under kompilering som vektorisering tar plats. Tiden det tar att kompilera testerna TestA, TestB och TestC förändras av att vektorisera.

Resultaten som visas i kompileringstids-diagrammet  visar att kompilerings-hastigheten förändras mellan -0.8\% och +1.1\%, och genomsnittligt +0.2\%.


Dessa resultat är intressanta inte bara för att prestandan av kompilerade program ökar, utan även för att det sker samtidigt som det tar mindre tid att utföra kompileringen.

Om vektoriseringen som presenteras här används i en mobiltelefon, hade batterianvändingen och tiden det tar att starta ett program kunnat minskas.

Denna forskning utfördes hos ARM Sweden på deras kontor i Lund.

\end{multicols}
\end{document}
